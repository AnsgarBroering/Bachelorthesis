% Ganz einfaches Template für das Proposal von studentischen Arbeiten
% von der Abteilung Digitalisierte Energiesysteme - 
% Carl von Ossietzky Universität Oldenburg
% Stand: 10.09.2020
%
\documentclass[a4paper, 11pt]{article}
% Zeilenabstand einstellen
\usepackage{setspace}
\setstretch{1.2}

%Serifenbehaftete Schrift für Überschriften
\setkomafont{sectioning}{\normalfont\normalcolor\bfseries}

%Serifenlose Schrift (beides einkommentieren)
%\renewcommand{\familydefault}{\sfdefault}
%\renewcommand{\sfdefault}{jkpss}

%Einstellungen für Farben
\definecolor{halfgray}{gray}{0.55}			% this color is necessary for the layout
%\definecolor{lightgray}{rgb}{0.7, 0.7, 0.7}
%\definecolor{lightergray}{rgb}{0.85, 0.85, 0.85}
%\definecolor{lightred}{rgb}{1.0, 0.5, 0.5}
\definecolor{linkcolor}{rgb}{0,0.3137,0.6078}	% LUH Blue
%\definecolor{linkcolor}{rgb}{0,0,0}				% Black


%Einzug neuer Absaetze
%\setlength\parindent{1cm}

%Einstellungen für Listings
\lstset{%
	basicstyle=\small\normalfont\hyphenchar\font\m@ne\@noligs,
	frame=single,
	captionpos=b,
	lineskip=-1pt,
	basewidth=0.5em,
	breaklines,
	aboveskip=20pt,
	belowskip=\medskipamount%
}

%Anklickbares Verzeichnis für PDF
\hypersetup{%
   pdftitle={\documenttitle},
   pdfsubject={\documentsubject},
   pdfkeywords={\documentkeywords},
   pdfauthor={\textcopyright~\documentauthor},
   colorlinks=true,
   linkcolor=linkcolor,
   citecolor=linkcolor,
   urlcolor=linkcolor,
   bookmarksnumbered,
   bookmarksopen=true,
   bookmarksopenlevel=1,
   %Test
   plainpages=false,
%   pdfpagelabels,
   hypertexnames=false % true,
%	linktocpage=true,
%	pdfstartview=FitV,
%	breaklinks=true,
%	pageanchor=true,
%	pageanchor=true,
%	pdfpagemode=UseOutlines,
%	pdfhighlight=/O,
}

% -----------------------------------------------------------------------------
% Microtype
% (siehe www.khirevich.com/latex/microtype)
%
\usepackage[
activate={true,nocompatibility},                                              % activate protrusion and expansion
final,                                                                        % enable microtype; use "draft" to disable
tracking=true,                                                                % activate tracking
kerning=true,                                                                 % activate kerning
spacing=true,                                                                 % activate spacing
factor=1100,                                                                  % add 10% to the protrusion amount (default is 1000)
stretch=10,                                                                   % reduce stretchability (default is 20)
shrink=10,                                                                    % reduce shrinkability (default is 20)
]{microtype}
%
\SetExtraKerning[unit=space]
{
	encoding={*},
	family={bch},
	series={*},
	size={footnotesize,small,normalsize}
}
{
	\textendash={400,400},                                                      % en-dash, add more space around it
	"28={ ,150},                                                             % left bracket, add space from right
	"29={150, },                                                             % right bracket, add space from left
	\textquotedblleft={ ,150},                                               % left quotation mark, space from right
	\textquotedblright={150, }                                               % right quotation mark, space from left
}
%
% -----------------------------------------------------------------------------


% -----------------------------------------------------------------------------
% Formatieren des Seitenlayouts:
% ************************************************************
% Fancy stuff (arsclassica)
% ************************************************************
\SetTracking[context=trackinglarge]{encoding = *}{160}                          % Definiere Kontext: 160er tracking
\SetTracking[context=trackingsmall]{encoding = *}{80}                           % Definiere Kontext: 80er tracking
%
\DeclareRobustCommand{\spacedallcaps}[1]{%                                      % Capitals (uppercase) mit 160er spacing
	\microtypesetup{expansion=false}%
	\fontfamily{jkpss}\lsstyle\microtypecontext{tracking=trackingsmall}%
	\MakeTextUppercase{#1}%
}
\DeclareRobustCommand{\spacedlowsmallcaps}[1]{%                                 % Capitals (lowercase) mit 80er spacing
	\microtypesetup{expansion=false}%
	\fontfamily{jkpss}\lsstyle\microtypecontext{tracking=trackingsmall}%
	\fontshape{sc}\selectfont\MakeTextLowercase{#1}%
}

% ************************************************************
% Headlines (arsclassica)
% ************************************************************
%\usepackage[automark]{scrpage2}
\clearmainofpairofpagestyles
\renewcommand{\chaptermark}[1]{%                                               % Kapitelname ohne Nummer, in small caps
	\markleft{{\spacedlowsmallcaps{#1}}}
	\markright{{\spacedlowsmallcaps{#1}}} %Also set on the right side til it is overwritten by sectionmark
}
%%
\renewcommand{\sectionmark}[1]{\markright{%                                     % Sectionname mit Nummer, in small caps
		{{\small\thesection} \spacedlowsmallcaps{#1}}%
}}

%
\lehead{\mbox{%                                                                 % Header oben links: Seitenzahl, vert. Linie, Kapitelname
		\llap{\small\pagemark\kern1em\color{halfgray}\vline}%
		\color{halfgray}\hspace{0.5em}\headmark\hfil%
}}
%
\rohead{\mbox{%                                                                 % Header oben rechts: Section, vert. Linie, Seitenzahl
		\hfil{\color{halfgray}\headmark\hspace{0.5em}}%
		\rlap{\small{\color{halfgray}\vline}\kern1em\pagemark}%
}}
%
\ofoot[\relax]{%                                                                % Keine Seitenzahlen in Fußzeile,
	\relax%                                                                       % auch nicht bei Kapitelanfang
}
%
\setkomafont{pageheadfoot}{%                                                    % Schriftart für Header-Text
	\normalfont\fontfamily{jkpss}\fontshape{sc}\selectfont%
}
%
\setkomafont{pagenumber}{%                                                      % Schriftart für Seitenzahlen
	\normalfont\small\fontfamily{jkpss}\selectfont%
}
%
%************************************************************
% Layout of the chapter-, section-, subsection-,
% subsubsection-, paragraph and description-headings (arsclassica)
%************************************************************

\newfont{\chapterNumber}{eurb10 scaled 5000}                                    % Lade skalierte Euler-Font für Kapitelnummer
%
\renewcommand*{\chapterformat}{%                                                % Große graue Euler-Kapitelnummer mit vert. Linie
	{%
		\color{halfgray}%
		\chapterNumber\thechapter%
	}%
	\hspace{15pt}\smash{\protect\rule[-7pt]{0.5pt}{42pt}}\hspace{15pt}%
}
%
\renewcommand*{\othersectionlevelsformat}[3]{%                                  % Stil der Abschnittsnummerierung
	\fontfamily{jkpss}\fontshape{sc}\selectfont%
	\MakeTextLowercase{%
		#3\autodot\enskip
	}%
}
%
\setkomafont{chapter}{%                                                         % Stil der Kapitelüberschriften
	\microtypecontext{tracking=trackinglarge}%
	\normalfont\Large\fontfamily{jkpss}%
	\lsstyle%
	%\MakeTextUppercase%
}
%
\setkomafont{section}{%                                                         % Stil der Sectionüberschriften
	\microtypecontext{tracking=trackingsmall}%
	\normalfont\Large\fontfamily{jkpss}\fontshape{sc}\selectfont%
	\lsstyle%
	% \lowercase%           <-- This would break UTF-8 chars...
}
%
\setkomafont{subsection}{%                                                      % Stil der Subsectionüberschriften
	\normalfont\normalsize\fontfamily{jkpss}\selectfont%
}
%
\setkomafont{subsubsection}{%                                                   % Stil der Subsubsectionüberschriften
	\normalfont\normalsize\itshape\fontfamily{jkpss}\selectfont%
}
%
\setkomafont{paragraph}{%                                                       % Stil der Paragraphüberschriften
	\normalfont\normalsize\bfseries\fontfamily{jkpss}\selectfont%
}


%% Variables
\newcommand{\Titel}{Example title} % Titel der Arbeit
\newcommand{\Author}{Example author} % Autor der Arbeit

%
\begin{document}
% -----------------------------------------------------------------------------
%               Titel
% -----------------------------------------------------------------------------
\Author \hfill \today\\
\newline
%
\begin{center}
	\iflanguage{ngerman}
		{\large{Proposal zur Masterarbeit}}
		{\large{Proposal for a master thesis}} \\
  	\vspace*{0.5cm}
  	\iflanguage{ngerman}
  		{\Large{\bf "`\Titel{}"'}}
  		{\Large{\bf ``\Titel{}''}}
\end{center}
%
\setlength{\parskip}{1.5ex plus0.5ex minus 0.5ex}

% -----------------------------------------------------------------------------
%\section{Einleitung}
\section{Introduction}% (one page)
\label{introduction}
The proposal should have a maximum length of five text pages (plus preliminary outline, schedule and literature). If it contains the "Forschungsseminar", it should be longer. It may also be less, if it contains everything necessary.
The proposal should include the motivation for the topic as well as relevant sources \cite{Crastan2008}.
If the following sections are to be referenced, this can be done with \cref{objective} or with \Cref{basics} depending on whether it should be written in upper or lower case (only relevant in English).


% -----------------------------------------------------------------------------
%\section{Zielsetzung}
\section{Objective}
\label{objective}
What should be achieved? Elaboration of the research questions. 

% -----------------------------------------------------------------------------
%\section{Grundlagen}
\section{Background}
\label{basics}
This section should present the basics that are relevant for working on the topic. Alternatively, the basics can be presented first and the objective can be derived from this.
% -----------------------------------------------------------------------------
%\section{Geplante Herangehensweise}
\section{Approach}
\label{approach}
Own planned contribution to the achievement of the objective.

% -----------------------------------------------------------------------------
%\section{Vorläufige Gliederung}
\section{Preliminary structure}
\label{structure}
This section should contain a draft of the chapter order and structure. Please also plan the number of pages for each chapter. This helps you to write a thesis with a normal size.
\begin{enumerate}
	\item Introduction (2)
	\begin{enumerate}
		\item Motivation
		\item ...
	\end{enumerate}
\end{enumerate}

% -----------------------------------------------------------------------------
%\section {Zeit- und Arbeitsplan}
\section {Time and work schedule}
\label{schedule}
You should also develop a time schedule for your planned approach. When planning your schedule, be sure to take into account other things that will take up your time during the thesis, such as lectures, exams, and extracurricular activities. Feel free to discuss this with your supervisor.
\begin{figure}[h]
	\centering
	\def\pgfcalendarmonthshortgerman#1{%
		\ifcase#1 Dez\or Jan\or Feb\or Mär\or Apr\or Mai\or Jun\or Jul\or Aug\or Sept\or Okt\or Nov\or Dez\fi%
	}

	\begin{ganttchart}[
		title/.append style={fill=black!10},
		x unit=1.8pt,
		time slot format=isodate,
		milestone/.append style={ultra thick}
		]{2018-05-21}{2018-11-28}
		\iflanguage{ngerman}
			{\gantttitlecalendar{year, month=shortgerman}}
			{\gantttitlecalendar{year, month=shortname}}
		\\
		\gantttitle{1}{7}
		\gantttitle{2}{7}
		\gantttitle{3}{7}
		\gantttitle{4}{7}
		\gantttitle{5}{7}
		\gantttitle{6}{7}
		\gantttitle{7}{7} 
		\gantttitle{8}{7}
		\gantttitle{9}{7}
		\gantttitle{10}{7}
		\gantttitle{11}{7}
		\gantttitle{12}{7}
		\gantttitle{13}{7}
		\gantttitle{14}{7}
		\gantttitle{15}{7}
		\gantttitle{16}{7}
		\gantttitle{17}{7}
		\gantttitle{18}{7}
		\gantttitle{19}{7}
		\gantttitle{20}{7}
		\gantttitle{21}{7}
		\gantttitle{22}{7}
		\gantttitle{23}{7}
		\gantttitle{24}{7}
		\gantttitle{25}{7}
		\gantttitle{26}{7}
		\gantttitle{27}{7}
		\gantttitle{}{3}\\
		\ganttmilestone{Registration of the thesis}{2018-05-21}\\
		\ganttbar{1. step}{2018-06-1}{2018-07-1}\\
		\ganttbar{2. step}{2018-07-1}{2018-09-1}\\
		\ganttbar{...}{2018-08-15}{2018-10-31}\\
		\ganttbar{Proofreading}{2018-11-1}{2018-11-19}\\
		\ganttmilestone{Submission}{2018-11-19}\\
		\ganttmilestone{Presentation}{2018-11-28}\\
	\end{ganttchart}
	\caption{\iflanguage{ngerman}{Zeitplan}{Time schedule}}
\end{figure}
\begin{itemize}
	\item 21.05.2018: Registration of the thesis (please allow sufficient time for the proposal)
	\item 01.07.2018: Completion step 1
	\item 01.08.2018: Completion step 2
	\item 01.09.2018: ...
	\item 20.11.2018: Submission
	\item 03.12.2018: Presentation
\end{itemize}

% -----------------------------------------------------------------------------
%               Literaturliste
% -----------------------------------------------------------------------------
\newpage
\bibliographystyle{bmc-des} % Style BST file (bmc-mathphys)
%\bibliographystyle{alpha}
%\bibliographystyle{abbrvdin}
\addcontentsline{toc}{chapter}{Literatur}
\bibliography{library}
\end{document}
